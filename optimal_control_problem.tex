\section{Прикладная задача поиска оптимального управления}
Задача глобальной оптимизации возникает при синтезе оптимальных с точки зрения некоторых
критериев управлений в линейных системах ОДУ. Если управление является линейной обратной
связью по состоянию, то система с управлением имеет вид:
\begin{displaymath}
    \dot x = (A+B_u\Theta)x + B_v v, x(0)=0,
\end{displaymath}
где  \(v(t)\in L_2\) --- некоторое возмущение.
Выходы системы описываются формулами \(z_k=(C_k+B_u\Theta),k=\overline{1,N}\).
Вляиние возмущения на \(k\)-й выход системы описывается критерием
\begin{displaymath}
  J_k(\Theta)=\sup_{v\in L_2} \frac{\max_{1\le i \le n_k} \sup_{t\ge 0}|z_k^{(i)}(\Theta,t)|}{||v||_2}.
\end{displaymath}

Нужно найти компоненты вектора \(\Theta\), минимизирующие один из критериев при
заданных ограничениях на другие:
\begin{displaymath}
   J_1(\Theta^*)=\min\{J_1(\Theta):J_k(\Theta)\le S_k,k=\overline{2,N}\}.
\end{displaymath}
В \cite{optControl} указан способ вычисления критериев, состоящий в решении СЛАУ,
а также даны решения рассматриваемой задачи в некоторых частных случаях. На
данных момент результаты, полученные в \cite{optControl} повторены с помощью системы Globalizer,
ведётся подготовка к решению более сложных задач из рассматриваемого класса, в которых
вычисление критериев --- трудоемкий процесс, требующий привлечения ресурсов вычислительного кластера.
